\chapter{SURF Site Preparation Activities}
\label{ch:fscf-site-prep}


%%%%%%%%%%%%%%%%%%%%%%%%%%%%%%%%%%%%%%%%%%%%%%%%%%%%%%%%%%%%%%%%%%%
\section{Overview}
\label{ch:fscf-site-prep-overview}

A number of activities at the SURF site that focus on maintenance or restoration of capabilities have been identified by the LBNF and SDSTA team as required for risk mitigation and/or construction preparation.   These tasks are planned to be completed before or during the LBNF Project and are included as part of the overall LBNF project cost and schedule due to their potential impact on the construction and operation of LBNF and DUNE.  They are, however, outside of the CD-3a request since they are not viewed as construction.
 
%%%%%%%%%%%%%%%%%%%%%%%%%%%%%%%%%%%%%%%%%%%%%%%%%%%%%%%%%%%%%%%%%%%
\section{Ross Shaft Rehabilitation}
\label{ch:fscf-site-prep-rossrehab}

The SDSTA has been in the process of rehabilitating the Ross Shaft since 2013.  This rehabilitation includes removal of all existing steel structural elements, installation of ground support (rock bolts and welded wire mesh), and installation of new structural steel.  Beginning in 2016, the funding for this project will be tied to the LBNF Project, but the actual implementation will not change.  When the rehabilitation completes in 2017, %the project will also include replacement of 
the \textit{cage}, or personnel conveyance, will be replaced with a new conveyance; this will restore the load capability of two decks, thereby improving personnel access.  Both of the \textit{skips}, which are buckets used to remove excavated material from the underground, will also be replaced with new ones to restore full functionality of the skips to remove rock for LBNF.
 
%%%%%%%%%%%%%%%%%%%%%%%%%%%%%%%%%%%%%%%%%%%%%%%%%%%%%%%%%%%%%%%%%%%
\section{Oro Hondo Fan Upgrade}
\label{ch:fscf-site-prep-orohondofan}

The primary ventilation fan for the entire underground facility was installed at the Oro Hondo Shaft in the mid-1980s to support mining efforts at various levels underground.  This fan was designed for much larger air volumes than will be necessary for LBNF/DUNE construction or operation, and uses less than 25\% of the 3000-HP motor's capacity.  The variable-speed drive for this fan is obsolete, with little to no  availability of parts.  To reduce the likelihood and impact of failure, this fan will be evaluated for efficiency and either a new motor/drive combination or a completely new fan will be installed prior to LBNF construction.
 
%%%%%%%%%%%%%%%%%%%%%%%%%%%%%%%%%%%%%%%%%%%%%%%%%%%%%%%%%%%%%%%%%%%
\section{Refuge Chamber Additions and Upgrades}
\label{ch:fscf-site-prep-refuge}

SURF currently has small refuge chambers at two of the four dewatering pump locations in the Ross Shaft, and a large refuge chamber near the Ross Shaft at the 4850L.  Refuge chambers are designed to provide food, water, breathable air, and sanitary facilities for individuals trapped underground.  In preparation for the LBNF Project, two additional pump room refuge chambers will be procured and installed, and the 4850L chamber will be modified to accommodate the additional capacity made possible by the Ross Shaft rehabilitation.  Underground occupancy is defined by the maximum number of individuals that can be transported to the surface in one hour.  With the improvements to the shaft, that number will approximately double (see Section~\ref{sec:fscf-und-fire}).
 
%%%%%%%%%%%%%%%%%%%%%%%%%%%%%%%%%%%%%%%%%%%%%%%%%%%%%%%%%%%%%%%%%%%
\section{Resupport of Drifts at the 4850L}
\label{ch:fscf-site-prep-ground}

The drifts (tunnels) connecting the Yates and Ross Shafts at the 4850L were excavated for mining purposes with ground support (rock bolts) installed only as deemed necessary at that time \fixme{give approx year (Josh)}.  This ground support was submerged in water when the mine shut down, accelerating corrosion.  The LBNF Project will provide supplies to re-support these drifts with full coverage in preparation for significantly increased traffic during construction and operation of LBNF and DUNE.
 
%%%%%%%%%%%%%%%%%%%%%%%%%%%%%%%%%%%%%%%%%%%%%%%%%%%%%%%%%%%%%%%%%%%
\section{Water Inflow Control}
\label{ch:fscf-site-prep-water}

While the SURF facility is generally very dry compared to most underground facilities, the vast expanse of the underground space both vertically and horizontally provides the opportunity for many small water inflows to aggregate, resulting in a total inflow of over 700 gallons per minute integrated over the 350+ miles of drifts.  The surface mining of the open cut aggravates this during large inflow events (rain or snow melt) by acting as a direct funnel to the upper levels of the facility.  This water migrates to either pump rooms or the \textit{pool}; at $\sim$1,000 feet below the 4850L, the pool provides enough reserve capacity to prevent flooding of the 4850L even if the dewatering system were shut down for nearly a year. 
A system of walls and boreholes controls the flow of water, keeping it away from the occupied footprint. These controls were installed throughout the 125-year history of mining, however, and cannot all be accessed for evaluation.  To prevent failure of an inaccessible control system, a project to capture and direct water from the upper levels to a known route is planned.
 
%%%%%%%%%%%%%%%%%%%%%%%%%%%%%%%%%%%%%%%%%%%%%%%%%%%%%%%%%%%%%%%%%%%
\section{Adit Repairs}
\label{ch:fscf-site-prep-adit}

A number of adits (tunnels that connect underground areas to the surface) will be rehabilitated to prevent their failure.  Specifically, two adits connecting at the 300L and two connecting at the \textit{tramway} level are included in the LBNF budget.  These adits support power, fiber, sewer, and water utilities, any of which could halt construction if a failure occurred.  Repairs are primarily focused on the first 60 -- 100 feet of the tunnels from the entrance at the exterior, with some repairs
 in the tunnels themselves.
 
%%%%%%%%%%%%%%%%%%%%%%%%%%%%%%%%%%%%%%%%%%%%%%%%%%%%%%%%%%%%%%%%%%%
\section{Ross Crusher Roof Reinforcement}
\label{ch:fscf-site-prep-rossroof}

As part of previous design efforts, many of the existing structures at SURF were evaluated for compatibility with current codes and standards.  The SDSTA has already repaired many of the substandard roofs throughout the facility, with the Ross Crusher building lagging due to lack of immediate use.  In preparation for use of this building by the LBNF project, the roof will be reinforced to ensure reliable support of snow loads.
 
%%%%%%%%%%%%%%%%%%%%%%%%%%%%%%%%%%%%%%%%%%%%%%%%%%%%%%%%%%%%%%%%%%%
\section{Hoist Motor Rebuilds}
\label{ch:fscf-site-prep-motor}

All of the hoist motors at both the Yates and Ross Shafts were evaluated by consultants through the SDSTA and found to require rebuilds for reliable operation.  The SDSTA started this process by rebuilding the motors at the Yates shaft, which currently provides primary access to the underground spaces while the Ross Shaft is rehabilitated.  After the shaft rehabilitation completes, but before excavation can commence, all of the motors at the Ross Shaft will be removed, cleaned, and re-insulated (effectively rebuilt) to reduce the risk of failure during the heavy construction utilization.
 
%%%%%%%%%%%%%%%%%%%%%%%%%%%%%%%%%%%%%%%%%%%%%%%%%%%%%%%%%%%%%%%%%%%
\section{Parking Lot Repairs}
\label{ch:fscf-site-prep-parking}

The surface facilities at SURF are located in steep and rugged terrain in Lead, SD.  The headframes and supporting buildings were constructed at the top of a hill, with cut-and-fill techniques used to provide flat and level areas for buildings, roads and parking lots beginning in the 1930s.  One of these parking lots, adjacent to the SURF administration building and near the Yates Shaft, experienced subsidence in the 1990s while the mine was still operational.  To temporarily manage this, a number of limestone blocks were placed in the area of concern.  In recent years, following significantly higher-than-normal precipitation, this area has again exhibited signs of movement.  To address this issue, a permanent retaining wall system has been designed by a local engineering firm contracted by the SDSTA.  The LBNF Project has included budget to implement this design to ensure that access to the Yates shaft is not compromised.
 