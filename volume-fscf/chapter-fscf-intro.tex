\chapter{Introduction to the LBNF Far Site Conventional Facilities}
\label{ch:fscf-intro}


%%%%%%%%%%%%%%%%%%%%%%%%%%%%%%%%%%%%%%%%%%%%%%%%%%%%%%%%%%%%%%%%%%%
\section{The LBNF Far Site Facilities Preliminary Design Report Volumes}
\label{sec:pdr-volumes-fscf}

The LBNF Far Site PDR ... \fixme{(need to make a common intro to pull from)}

%%%%%%%%%%%%%%%%%%%%%%%%%%%%%%%%%%
%\subsection{A Roadmap of the PDR}
%%%%%%%%%%%%%%%%%%%%%%%%%%%%%%%%%%
\subsection{A Roadmap of the PDR}
%\label{sec:pdr-volumes-roadmap}

text 
\begin{itemize}
 \item \volintro{} provides ... \fixme{}
 \item \volfscf{} provides ... \fixme{}
 \item \volcryo{} provides ... \fixme{}
\end{itemize}

More detailed information for each of these volumes is provided in a set of annexes listed on the CD-3a Reports and Documents page \fixme{give url}.


\fixme{need to update roadmap}

%%%%%%%%%%%%%%%%%%%%%%%%%%%%%%%%%%
\subsection{About this Volume}
\label{sec:pdr-volumes-vol1}

PDR \volfscf{} describes the facilities at the Sanford Underground Research Facility (SURF) 
 at Lead, South Dakota, the far site location identified for the LBNF project.
 The remainder of this chapter provides a short introduction to these facilities.  The management structure for LBNF is summarized in Chapter 2 of \volintro.
Chapter~\ref{ch:fscf-site-cond} describes the existing site conditions at the (SURF). 
Chapter~\ref{ch:fscf-surf-facil} describes the existing and planned surface buildings that will support the the DUNE far detector, planned for installation at the 4850L. 
Chapter~\ref{ch:fscf-excav} discusses the planned underground excavation, and 
Chapter~\ref{ch:fscf-und-infra}  describes the underground infrastructure that will directly interface to the DUNE far detector modules.

%%%%%%%%%%%%%%%%%%%%%%%%%%%%%%%%%%
\subsection{Introduction to the Far Site Conventional Facilities}
\label{sec:fs-facil-cf}

\fixme{this is taken from cdr vol 3, 3.1 Overview}

This volume presents the scope and necessary steps required to develop the LBNF Far Site Conventional Facilities (FSCF) at SURF. The key element of the FSCF is the underground space required to install and support the operations of the DUNE far detector. An overview of the 4850L at SURF where the underground facilities will be developed is shown in Figure~\ref{fig:fs-main-components}.

\begin{cdrfigure}[Far Site: Main components at the 4850 level]{fs-main-components}{Far Site: Main components at the 4850 level (underground)}
%\includegraphics[width=0.8\textwidth]{fs-main-components}
\end{cdrfigure}

While the SURF site meets many requirements from the geological, scientific and engineering standpoint, significant work is required to provide adequate space and infrastructure support needed for the experiment's installation and operation. The present and future state of the site, evaluation and assessments of the facilities and the associated provisioning of infrastructure such as power, water, plumbing, ventilation, etc., as well as safety measures and planned steps to develop the surface and underground structures are described at a high level.  

For more information on FSCF, refer to \fixme{reference the arup doc}

