\chapter{Introduction to the LBNF Far Site Facilities}
\label{ch:intro-intro}

%%%%%%%%%%%%%%%%%%%%%%%%%%%%%%%%%%%%%%%%%%%%%%%%%%%%%%%%%%%%%%%%%%%
\section{The LBNF Far Site Facilities in the Context of LBNF/DUNE}
\label{sec:fs-facil-context}

The global neutrino physics community is developing a multi-decade physics program to measure 
unknown parameters of the Standard Model of particle physics and search for new phenomena. The 
program will be carried out as an international, leading-edge, dual-site experiment for neutrino science 
and proton decay studies, which is known as the Deep Underground Neutrino Experiment (DUNE). The 
detectors for this experiment will be designed, built, commissioned and operated by the international 
DUNE Collaboration. The facility required to support this experiment, the Long-Baseline Neutrino Facility 
(LBNF), is hosted by Fermilab and its design and construction is organized as a DOE/Fermilab project 
incorporating international partners. Together LBNF and DUNE will comprise the world's highest-intensity 
neutrino beam at Fermilab, in Batavia, IL, a high-precision near detector on the Fermilab site, a massive 
liquid argon time-projection chamber (LArTPC) far detector installed deep underground at the Sanford 
Underground Research Facility (SURF) 1300 km away in Lead, SD, and all of the conventional and technical 
facilities necessary to support the beamline and detector systems.

The strategy for executing the experimental program was presented in the LBNF/DUNE Conceptual Design Report (CDR) \fixme{add ref}. It  
has been developed to meet the requirements set out in the P5 report [1] \fixme{add ref} and takes into 
account the recommendations of the European Strategy for Particle Physics [2]. \fixme{add ref} It adopts a 
model in which U.S. and international funding agencies share costs on the DUNE detectors, and CERN and 
other participants provide in-kind contributions to the supporting infrastructure  of LBNF. LBNF and DUNE 
will be tightly coordinated as DUNE collaborators design the detectors and infrastructure that will carry out 
the scientific program.

The scope of LBNF is
\begin{itemize}
 \item an intense neutrino beam aimed at the far site
 \item conventional facilities at both the near and far sites
 \item cryogenics infrastructure to support the DUNE detector at the far site
The DUNE detectors include
 \item a beamline measurement system
\end{itemize}
The DUNE detectors include
\begin{itemize}
 \item a high-performance neutrino detector located a few hundred meters downstream of the neutrino source
 \item a massive liquid argon time-projection chamber (LArTPC) neutrino detector located deep underground at the far site
\end{itemize}


With the facilities provided by LBNF and the detectors provided by DUNE, the DUNE Collaboration proposes to mount a focused attack on the puzzle of neutrinos with broad sensitivity to neutrino oscillation parameters in a single experiment. The focus of the scientific program is the determination of the neutrino mass hierarchy and the explicit demonstration of leptonic CP violation, if it exists, by precisely measuring differences between the oscillations of muon-type neutrinos and antineutrinos into electron-type neutrinos and antineutrinos, respectively. Siting the far detector deep underground will provide exciting additional research opportunities in nucleon decay, studies utilizing atmospheric neutrinos, and neutrino astrophysics, including measurements of neutrinos from a core-collapse supernova should such an event occur in our galaxy during the experiment's lifetime.


%%%%%%%%%%%%%%%%%%%%%%%%%%%%%%%%%%%%%%%%%%%%%%%%%%%%%%%%%%%%%%%%%%%
\section{The LBNF Far Site Facilities Preliminary Design Report Volumes}
\label{sec:pdr-volumes}

The LBNF Far Site PDR ...


%%%%%%%%%%%%%%%%%%%%%%%%%%%%%%%%%%
\subsection{A Roadmap of the PDR}
\label{sec:pdr-volumes-roadmap}

text 
\begin{itemize}
 \item \volintro{} provides ... \fixme{}
 \item \volfscf{} provides ... \fixme{}
 \item \volcryo{} provides ... \fixme{}
\end{itemize}

More detailed information for each of these volumes is provided in a set of annexes listed on the CD-3a Reports and Documents page \fixme{give url}.

%%%%%%%%%%%%%%%%%%%%%%%%%%%%%%%%%%
\subsection{About this Volume}
\label{sec:pdr-volumes-vol1}

text

%%%%%%%%%%%%%%%%%%%%%%%%%%%%%%%%%%%%%%%%%%%%%%%%%%%%%%%%%%%%%%%%%%%
\section{Introduction to the LBNF Far Site Facilities}
\label{sec:fs-facil-intro}

\fixme{from CDR vol 3 sec 1.3.1}

The scope of LBNF at SURF includes both conventional facilities (CF) and cryogenics infrastructure to support the DUNE far detector. Figure 3 1 \fixme{} shows the layout of the underground caverns that house the detector modules with a separate cavern to house utilities and cryogenics systems. The requirements derive from DUNE Collaboration science requirements [5] \fixme{}, which drive the space and functions necessary to construct and operate the far detector.  Environment, Safety and Health (ES\&H) and facility operations (programmatic) requirements also provide input to the design. The far detector is modularized into four 10-kt fiducial mass detectors. The caverns and the services to the caverns will be similar to one another as much as possible to enable repetitive designs for efficiency in design and construction as well as operation. 



%%%%%%%%%%%%%%%%%%%%%%%%%%%%%%%%%%
\subsection{The Far Site Conventional Facilities}
\label{sec:fs-facil-cf}

The scope of the conventional facilities includes design and construction for facilities on the surface and underground. The underground conventional facilities includes new excavated spaces at the 4850L for the detector, utility spaces for experimental equipment, utility spaces for facility equipment, drifts for access, as well as construction-required spaces. Underground infrastructure provided by CF for the experiment includes power to experimental equipment, cooling systems, and cyberinfrastructure. Underground infrastructure necessary for the facility includes domestic (potable) water, industrial water for process and fire suppression, fire detection and alarm, normal and standby power systems, a sump pump drainage system for native and leak water around the detector, water drainage to the facility-wide pump discharge system, and cyberinfrastructure for communications and security.  In addition to providing new spaces and infrastructure underground, CF enlarges and provides infrastructure in some existing spaces for use, such as the access drifts from the Ross Shaft to the new caverns. New piping is provided in the shaft for cryogens (gas argon transfer line and the compressor suction and discharge lines) and domestic water as well as power conduits for normal and standby power and cyberinfrastructure. 

The existing Sanford Laboratory has many surface buildings and utilities, some of which will be utilized for LBNF. The scope of the above ground CF includes only that work necessary for LBNF, and not for the general rehabilitation of buildings on the site, which remains the responsibility of the Sanford Laboratory. Electrical substations and distribution will be upgraded to increase power and provide standby capability for life safety. Additional surface scope includes a small control room in an existing building and a new building to support cryogen transfer from the surface to the underground near the existing Ross Shaft.
To reduce risk of failure of essential but aging support equipment during the construction and installation period, several SURF infrastructure operations/maintenance activities are included as early activities in LBNF. These include completion of the Ross Shaft rehabilitation, rebuilding of hoist motors, and replacement of the Oro Hondo fan; if not addressed, this aging infrastructure could limit or stop access to the underground if equipment failed. 



%%%%%%%%%%%%%%%%%%%%%%%%%%%%%%%%%%
\subsection{The Far Site Cryogenic Infrastructure}
\label{sec:fs-facil-cryo}

The scope of the LBNF Cryogenics Infrastructure includes the design, fabrication, and installation of four cryostats to contain the liquid argon (LAr) and the detector components and a comprehensive cryogenics system that meets the performance requirements for purging, cooling down and filling the cryostats, achieving and maintaining the LAr temperature, and purifying the LAr outside the cryostats. 

Each cryostat is composed of a free-standing steel-framed structure with a membrane cryostat vessel installed inside, to be constructed in one of the four excavated detector pits. The cryostat is designed to hold a total LAr mass capacity of 17.1 kt. Each membrane tank cryostat has a stainless-steel liner as part of a full membrane system to provide full containment of the liquid cryogen. The hydrostatic pressure loading of the liquid cryogen is transmitted through rigid foam insulation to the surrounding structural steel frame which provides external support for the membrane. All penetrations into the interior of the cryostat will be made through the top plate to minimize the potential for leaks with the exception of the sidewall penetration for connection to the LAr recirculation system.

Cryogenics system components are located both on the surface and within the cavern. The cryogen receiving station is located on the surface near the Ross Shaft to allow for receipt of LAr deliveries for the initial filling period, as well as a buffer volume to accept liquid argon during the extended fill period. A large vaporizer for the nitrogen circuit feeds gas to one of four compressors located in the Cryogenics Compressor Building; the compressor discharges high pressure nitrogen gas to pipes in the Ross shaft. The compressors are located on the surface because the electrical power requirement and cooling requirement is much less than for similar equipment at the 4850L. 

Equipment at the 4850L includes the nitrogen refrigerator, liquid nitrogen vessels, argon condensers, external liquid argon recirculation pumps, and filtration equipment. Filling each cryostat with LAr in a reasonable period of time is a driving factor for the refrigerator and condenser sizing. Each cryostat will have its own overpressure protection system, argon recondensers, and argon-purifying equipment, located in the Central Utility Cavern. Recirculation pumps will be placed outside and adjacent to each cryostat to circulate liquid from the bottom of the tank through the purifier.



%%%%%%%%%%%%%%%%%%%%%%%%%%%%%%%%%%%%%%%%%%%%%%%%%%%%%%%%%%%%%%%%%%%
%%%%%%%%%%%%%%%%%%%%%%%%%%%%%%%%%%%%%%%%%%%%%%%%%%%%%%%%%%%%%%%%%%%
Reference stuff:

Refer to figure as Figure~\ref{fig:label} and uncomment once there's a real filename to enter.

\begin{cdrfigure}[short]{label}{long}
%\includegraphics[width=0.8\textwidth]{filename}
\end{cdrfigure}

Here's a sample table; refer to it as Table~\ref{tab:label}.

\begin{cdrtable}[short]{lcc}{label}{long }
Physics milestone & Exposure \ktMWyr{} & Exposure \ktMWyr{}\\ \rowtitlestyle
  & (reference beam) & (optimized beam) \\ \toprowrule 
  $1^\circ$ $\theta_{23}$ resolution ($\theta_{23} = 42^\circ$) & 70  &  45\\ \colhline
  CPV at $3\sigma$ ($\delta_{\rm CP} = +\pi/2$)  & 70 & 60 \\ \colhline
  CPV at $3\sigma$ 75\% of \deltacp & 1320 & 850\\ 
\end{cdrtable}





